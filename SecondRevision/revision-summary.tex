


\documentclass[10pt,a4paper,oneside,reqno]{amsart}
\usepackage[parfill]{parskip} 


\usepackage{amsmath,geometry,amssymb,mathpartir,mathtools,latexsym,amsthm,enumerate,leftidx,tikz,url}
 
\usepackage[all]{xy}
\newcommand{\xycenter}[1]{\vcenter{\hbox{\xymatrix{#1}}}}
\SelectTips{cm}{}
\input{prooftree}

% Numberings 
\setcounter{tocdepth}{1}
\numberwithin{equation}{section}

% Table of contents
\makeatletter
\def\@tocline#1#2#3#4#5#6#7{\relax
\ifnum #1>\c@tocdepth % then omit
  \else 
    \par \addpenalty\@secpenalty\addvspace{#2}% 
\begingroup \hyphenpenalty\@M
    \@ifempty{#4}{%
      \@tempdima\csname r@tocindent\number#1\endcsname\relax
 }{%
   \@tempdima#4\relax
 }%
 \parindent\z@ \leftskip#3\relax \advance\leftskip\@tempdima\relax
 \rightskip\@pnumwidth plus4em \parfillskip-\@pnumwidth
 #5\leavevmode\hskip-\@tempdima #6\nobreak\relax
 \ifnum#1<0\hfill\else\dotfill\fi\hbox to\@pnumwidth{\@tocpagenum{#7}}\par
 \nobreak
 \endgroup
  \fi}
\makeatother

\def\noqed{\renewcommand{\qedsymbol}{}}

% Theorem

\newtheoremstyle{mythm}% 
{10pt}% Space above 
{}% Space below 
{\itshape}% Body font 
{}% Indent amount 
{\bfseries}%  Theorem head font 
{.}% Punctuation after theorem head 
{.5em}% Space after theorem head 
{}% 

\newtheoremstyle{mydef}% 
{10pt}% Space above 
{3pt}% Space below 
{}% Body font 
{}% Indent amount 
{\bfseries}%  Theorem head font 
{.}% Punctuation after theorem head 
{.5em}% Space after theorem head 
{}% 

\newtheoremstyle{myrmk}% 
{10pt}% Space above 
{3pt}% Space below 
{}% Body font 
{}% Indent amount 
{\itshape}%  Theorem head font 
{.}% Punctuation after theorem head 
{.5em}% Space after theorem head 
{}% 

\theoremstyle{mythm}
\newtheorem{theorem}{Theorem}[section]
\newtheorem*{theorem*}{Theorem}
\newtheorem{lemma}[theorem]{Lemma} 
\newtheorem{proposition}[theorem]{Proposition} 
\newtheorem{corollary}[theorem]{Corollary}  
\newtheorem{apptheorem}{Theorem}
\newtheorem{atheorem}{Theorem}
\renewcommand*{\theatheorem}{\Alph{atheorem}}
\theoremstyle{mydef}
\newtheorem{definition}[theorem]{Definition}	
\newtheorem*{definition*}{Definition}	
\theoremstyle{myrmk}
\newtheorem{remark}[theorem]{Remark} 
\newtheorem{remarks}[theorem]{Remarks} 
\newtheorem*{remark*}{Remark} 
\newtheorem*{remarks*}{Remarks} 
\newtheorem{example}[theorem]{Example}
\newtheorem{examples}[theorem]{Examples}
\newtheorem*{example*}{Example}
\newtheorem*{examples*}{Examples}

% Text commands
\newcommand{\ie}{\text{i.e.\ }}
\newcommand{\eg}{\text{e.g.}}
\newcommand{\resp}{\text{resp.\ }}
\newcommand{\myemph}[1]{\textit{#1}}
\newcommand{\by}[1]{\quad&&\text{by {$#1$}}}

% Judgemental equality, path equality, homotopy, definitions
\newcommand{\deq}{=}
\newcommand{\peq}{\Rightarrow}
\newcommand{\ho}{\sim}
\newcommand{\defeq}{=_{\mathrm{def}}}
\newcommand{\co}{\,{:}\,}
\newcommand{\type}{\mathsf{type}}
\newcommand{\com}{\circ}
\newcommand{\iso}{\cong}

% Composition and identiies
\newcommand{\idfun}[1]{\mathsf{id}_{#1}}
% \newcommand{\comp}{\circ}
\newcommand{\ct}{\cdot}

% Names for type theories
\newcommand{\Hint}{\mathcal{H}}
\newcommand{\Hext}{\mathcal{H}^{\mathrm{ext}}}
\newcommand{\Mint}{\mathcam{M}^{\mathrm{ext}}}
\newcommand{\Mext}{\mathcal{M}}
\newcommand{\UA}{\mathrm{UA}}

% General notions

\newcommand{\hfiber}{\mathsf{hfiber}}
\newcommand{\iscontr}{\mathsf{iscontr}}
\newcommand{\isntype}[1]{\mathsf{is}\text{-}\mathsf{#1}\text{-}\mathsf{type}}
\newcommand{\isprop}{\mathsf{isprop}}
\newcommand{\isequiv}{\mathsf{isequiv}}
\newcommand{\isind}{\mathsf{isind}}
\newcommand{\isbipind}{\mathsf{isind}}
\newcommand{\isalgind}{\mathsf{isind}}
\newcommand{\ishinit}{\mathsf{ishinit}}
\newcommand{\isbiphinit}{\mathsf{ishinit}}
\newcommand{\isalghinit}{\mathsf{ishinit}}
\newcommand{\trans}{\mathsf{tr}}
\newcommand{\Hot}{\mathsf{Hot}}
\newcommand{\Eq}{\mathsf{Equiv}}

% Extensionality

\newcommand{\ext}{\mathsf{ext}}
\renewcommand{\int}{\mathsf{int}}
\newcommand{\idtodpair}{\ext}
\newcommand{\happly}{\exti}
\newcommand{\funext}{\int}       
\newcommand{\idtoeq}{\ext}

% Bool

\newcommand{\Bool}{\mathsf{Bool}}
\newcommand{\true}{1}
\newcommand{\false}{0}
\newcommand{\one}{\mathsf{1}}
\newcommand{\zero}{\mathsf{0}}
\newcommand{\boolind}{\mathsf{boolelim}}
\newcommand{\boolrec}{\mathsf{boolrec}}

% Nat

\newcommand{\nat}{\ensuremath{\mathbb{N}}} 
\newcommand{\suc}{\mathsf{succ}}

% Id-types

\newcommand{\Id}{\mathsf{Id}}
\newcommand{\id}[1]{\Id_{#1}}
\newcommand{\refl}{\mathsf{refl}}
\newcommand{\idrec}{\mathsf{J}}

% Pi-types

\newcommand{\prd}[1]{\Pi_{#1}}
\newcommand{\lam}[1]{\lambda_{#1}}   
\newcommand{\app}{\mathsf{app}}

% Sigma-types

\newcommand{\sm}[1]{\Sigma_{#1}}    
\newcommand{\pair}{\mathsf{pair}}
\newcommand{\mysplit}{\mathsf{split}}
\newcommand{\fst}{\mathsf{fst}}
\newcommand{\snd}{\mathsf{snd}}

% W-types

\newcommand{\W}{\mathrm{W}}
\newcommand{\wsup}{\mathsf{sup}}
\newcommand{\wrec}{\mathsf{wrec}}
\newcommand{\wind}{\mathsf{elim}}
\newcommand{\wcomp}{\mathsf{wcomp}}

% Universes

\newcommand{\U}{\mathsf{U}}

% Bipointed types

\newcommand{\ind}{\mathsf{ind}}
\newcommand{\Hom}{\mathsf{Hom}}
\newcommand{\Bip}{\mathsf{Bip}}
\newcommand{\BipHom}{\mathsf{Bip}}
\newcommand{\BipHot}{\mathsf{BipHot}}
\newcommand{\FibBip}{\mathsf{FibBip}}
\newcommand{\BipSec}{\mathsf{BipSec}}
\newcommand{\isbipequiv}{\mathsf{isbipequiv}}
\newcommand{\BipEquiv}{\mathsf{BipEquiv}}
\newcommand{\HoSec}{\mathsf{BipSecHo}}
\newcommand{\elim}{\mathsf{elim}}
\newcommand{\comp}{\mathsf{comp}}
\newcommand{\rec}{\mathsf{rec}}
\newcommand{\Tot}{\mathsf{Tot}}

% P-algebras

\newcommand{\Palg}{\mathsf{Alg}}
\renewcommand{\sup}{\mathrm{sup}}
\newcommand{\isalgequiv}{\mathsf{isalgequiv}}
\newcommand{\AlgEquiv}{\mathsf{AlgEquiv}}
\newcommand{\AlgHot}{\mathsf{AlgHot}}
\newcommand{\FibAlgHot}{\mathsf{FibAlgHot}}
\newcommand{\FibPalg}{\mathsf{FibAlg}}
\newcommand{\PalgSec}{\mathsf{AlgSec}}
\newcommand{\AlgSecHot}{\mathsf{AlgSecHot}}



% DOCUMENT 

\begin{document}

\title[]{Summary of revisions}
\author[S. Awodey]{STEVE AWODEY}

\author[N. Gambino]{NICOLA GAMBINO}

\author[K. Sojakova]{KRISTINA SOJAKOVA}

\date{\today}

\maketitle

\section{Main referee comments}

\subsection*{Lack of $\eta$-rule for $\Sigma$-types} \hfill 

-- \emph{Why are you not assuming eta-rules for Sigma-types? Do you have models in mind where this is not validated? (It should be true in e.g. the cubical sets model, right?)}

-- \emph{How do you know that you are not using eta-rules in the formalisation? The HoTT library assumes eta for Sigma: \url{http://hott.github.io/HoTT/coqdoc-html/HoTT.types.Sigma.html}\#\url{eta_sigma}}

-- \emph{If you did use eta for Sigma, your polynomial pseudofunctors would be strict functors, and your life would be easier (for instance, it would mean that in diagram 4.4, we could define $\bar{1} = \refl$, instead of $\bar{1} = ap_{- \circ sup} q$, where $q : P(id) = id$). So unless you have good reasons not to, I would suggest to adopt eta-rules also for Sigma.}

-- \emph{Having said that, "pretending" that P is a pseudofunctor only is good for paving the way towards dealing with given unknown (non-polynomial) functors, where we will have no choice but to also ask for proofs of pseudofunctoriality.}


We prefer to avoid the assumption of the judgemental $\eta$-rule for $\Sigma$-types in the paper, since we would like keep the type theories in which we work as close as possible to Martin-L\"of type theories, where judgemental $\eta$-rules for inductive types are  not assumed.

However, we have added a remark (Remark 4.1 of the revised version) to explain that some proofs would simplify if the judgemental $\eta$-rule for $\Sigma$-types is assumed (along the lines suggested by the referee).

We have also rewritten the paragraph `Formalization' of the Introduction to clarify the relationship between the results in the paper and those in the formalisation files. The files have been written for compilation with the current version of Coq and therefore they use the judgemental $\eta$-rule for $\Sigma$-types (precisely as explained in Remark 4.1), but the results in the paper should be formalizable also in other proof-checkers
where that rule is not assumed.


\subsection*{Similarities(?) between your coherence rules and triangular laws for adjunctions}  \hfill

-- \emph{In several places, you point out that your coherence rules are reminiscent of one of the triangular laws for an adjunction. I'm not sure if I see this... Yes, there is the coincidence that two different things are traditionally called eta, and if (on e.g. page 17) one chooses f = id, then the "type" of eta is at least the same as the type of the unit of an adjunction, and the triangle really has id on one side, but why does it make sense to think of the
 beta-equality as a counit? And what about arbitrary f?}

- \emph{If you have more insight, I would appreciate if you could expand on this point; if not, I would recommend to remove it from the paper.}


We have removed the confusing remarks from the paper. We left the diagrams and rephrased the surrounding text.  See the text before the proof of Proposition 3.3, after the proof of Proposition 3.8, after the proof of Proposition 5.8  for details.

\section{Referee comments on the Coq formalization}

\emph{It would be nice if you could write some more comments and/or change theorem names in the Coq
source files linking the formalization with the paper -- the source files are quite terse to read at the moment.} 

We have added comments to all definitions and lemmas, linking the formalization with the paper and explaining, where applicable, any differences from the paper version. 

\emph{Some lemmas are (harmlessly) mislabelled in the formalisation:
\begin{itemize}
\item  two.v:
 \begin{itemize}
 \item isind2hasfibetacoh is not all of Prop 3.3 as claimed.
 \item ishinitEQisrechasetacoh is not Prop 3.8.
 \item ishinitEQeq2hinit is not Cor 3.12
\end{itemize}
\item  w.v:
\begin{itemize}
 \item "(** Lemma 5.8 **)" should be Lemma 4.8
 \item isind2hasfibetacoh is not all of Prop 5.4 as claimed.
 \item ishinitEQisrechasetacoh is not Prop 5.8.
 \item ishinitEQeq2hinit is not Cor 5.12 \\ 
\end{itemize}
\end{itemize}}

We fixed all of the above.

\emph{In the statement of Lambek's Lemma (Lemma lambek), wouldn't things be nicer if you actually used Sigma-types when defining algebras and fibered algebras?}

We prefer to use the uncurried form of the structure map for $P$-algebras 
since it is more in line with the dependent form of the structure map, which does not use $\Sigma$-types,
 and because many of the definitions (and some proofs) in the formalization become simpler since we do not have to perform pattern matching on an argument of a $\Sigma$-type. 

\section{Minor referee comments}


-- \emph{page 5, line 40: when I reached this line, I was wondering if you would need the "non-naive" version of function extensionality which states that happly is an equivalence. Maybe put in a forward reference to where you say that this is derivable from the naive version on page 8? Also, how about including a reference (e.g. to
 the HoTT book, or Voevodsky) for naive funext implies funext there?}

We have added a forward reference below equation (1.5).  

-- \emph{page 21, line 35: as a matter of style, I would prefer it if you just combined Corollary 3.11 and Theorem 3.10; the proof of Corollary 3.11 would become the first sentence of the proof of Theorem 3.10, and then nothing else needs to change. This would also help prevent the rash conclusion that there probably is no
 equivalence isind(A) <~> ishinit(A) when reaching Theorem 3.10, because otherwise why didn't you say so?}  

The two results have been merged into what is now Theorem 3.9. 

-- \emph{page 23, line 52: you are creating an identity proof between two pairs by giving a pair of identity proofs; I don't think you have introduced this notation -- probably easiest (but noisier) to use your int function.} 

We have added the use of $\mathsf{int}$, see equation (4.2) on page 24. 

-- \emph{page 27, lines 12--30: this feels a bit messy, and some definitions come completely out of the blue (for instance, it would be good if $e_f$ could be motivated and not just given). I think a second pass through this paragraph would be good.} 

The paragraph has been completely rephased: it now starts at the bottom of page 27 and the explanation of the definition of $e_f$ ends after the diagram in the middle of page 28.

-- \emph{page 28, line 43: spell out how Lemma 4.4 is a special case of Lemma 4.8 -- I don't think it is completely trivial or immediately obvious.} 

We gave a direct proof of (what is now) Lemma 4.5. The derivation of it from (what is now) Lemma 4.9 is indeed rather laborious and would require adding further material, not needed elsewhere. 

-- \emph{page 34, line 48: say that you get the special case by considering a constant family $E(z) = D$.} 

This has been done, see proof of Theorem 5.9. 

-- \emph{page 36, line 50: same comment as page 21, line 35.} 

We merged the two results into Theorem 5.9.




\subsection*{Typos} \hfill 

All typos have been fixed. 

\section{Conclusion}

As requested by the Editor, we have also added a substantial concluding section. The main results are indicated, as are several topics for future investigation.


\end{document}  
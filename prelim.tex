The general topic of Homotopy Type Theory is concerned with the study of the constructive type theories of Martin-L\"of under their new interpretation into abstract homotopy theory and higher-dimensional category theory. Martin-L\"of type theories are foundational systems which have been used to formalize large parts of constructive mathematics, and also for the development of high-level programming languages~\cite{MartinLofP:conmcp}.  They are prized for their combination of expressive strength and desirable proof-theoretic properties.  One aspect of these type theories that has led to special difficulties in providing semantics is the intensional character of equality.  In recent work \cite{AwodeyS:homtmi,VoevodskyV:notts,vandenBergB:topsmi,AwodeyS:typth}, it has emerged that the topological notion of \emph{homotopy} provides an adequate basis for the semantics of intensionality.  This extends the paradigm of computability as continuity, familiar from domain theory, beyond the simply-typed 
$\lambda$-calculus to dependently-typed theories involving:\begin{enumerate}[(i)]
\item dependent sums $(\Sigma x\colon\!{A})B(x)$ and dependent products $(\Pi x\colon\!{A})B(x)$, modelled respectively by the total space and the space of sections of the fibration modelling the dependency of $B(x)$ over $ x : A$; \item
and, crucially, including the identity type constructor~$\id{A}(a,b)$, interpreted as the space of all \emph{paths} in~$A$ between points~$a$ and~$b$. \end{enumerate}

In the present work, we build on this homotopical interpretation to study inductive types, such as the natural numbers, Booleans, lists, and W-types. Within extensional type theories, W-types can be used to  provide a constructive counterpart of the classical notion of a well-ordering~\cite{MartinLofP:inttt} and to uniformly define a variety of inductive types~\cite{DybjerP:repids}.
However, most programming languages and proof assistants, such as Coq~\cite{BertotY:inttpp}, Agda~\cite{NorellU:towppl} and Epigram~\cite{McBrideC:viefl} use schematic inductive definitions~\cite{CoquandT:inddt,PaulinMorhringC:inddsc} rather than W-types to define inductive types.  This is due in part to the practical convenience of the schematic approach, but it is also a matter of necessity; these systems are based on intensional rather than extensional type theories, and in the intensional theory the usual reductions of inductive types to W-types fail~\cite{DybjerP:repids,McBrideC:wtygnb}.
Nonetheless, W-types retain great importance from a theoretical perspective, since they allow us to internalize in type theory arguments about inductive types. Furthermore, as we will see in Section~\ref{section:intW}, a limited form of extensionality licensed by the homotopical interpretation suffices to develop the theory of W-types in a satisfactory way. In particular, we shall make use of ideas from higher category theory and homotopy theory to understand W-types as ``homotopy-initial" algebras of an appropriate kind.

%%%%%%%%%%%%%%%%%%%%%%%
\subsection{Extensional vs.\ intensional type theories}

\noindent We work here with type theories that have the four standard forms of judgement
\[
A : \type \, , \quad A = B : \type \, , \quad   a : A \, , \quad a = b : A \, . 
\]
We refer to the equality relation in these judgements as \emph{definitional equality}, 
which should be contrasted with the notion of \emph{propositional equality}
recalled below. 
Such a judgement $J$ can be made also relative to a \emph{context}~$\Gamma$ of variable declarations, a situation that we indicate by writing~$\Gamma \vdash J$. When stating deduction
rules we make use of standard conventions to simplify the exposition, such as omitting the mention
of a context that is common to premisses and conclusions of the rule.
The rules for identity types in intensional type theories are given in~\cite[Section~5.5]{NordstromB:marltt}. We recall them here in a slighly different, but equivalent, formulation.

\begin{itemize}
\item $\Id$-formation rule.
\[
\begin{prooftree}
A :  \type \quad 
a :  A  \quad
b :  A 
\justifies
 \id{A}(a,b) :  \type
 \end{prooftree}
\]
\item $\Id$-introduction rule.
\[
\begin{prooftree}
a :  A 
\justifies
 \refl(a) :  \id{A}(a,a)
 \end{prooftree} 
\]
\item $\Id$-elimination rule.
\[
\begin{prooftree}
\begin{array}{l} 
x, y :  A, u :  \id{A}(x,y) \vdash C(x,y,u) :  \type \\
 x :  A \vdash  c(x) :  C(x,x,\refl(x))  
 \end{array}
\justifies
x, y :  A, u :  \id{A}(x,y) \vdash  \idrec(x,y,u,c) :  C(x,y,u)
\end{prooftree}
\]
\item $\Id$-computation rule.
\[
\begin{prooftree}
\begin{array}{l} 
x, y :  A, u :  \id{A}(x,y) \vdash C(x,y,u) :  \type \\
 x :  A \vdash  c(x) :  C(x,x,\refl(x)) 
 \end{array}
 \justifies
x :  A \vdash \idrec(x,x,\refl(x), c) = c(x) :  C(x, x, \refl(x)) \, .
\end{prooftree}
\]
\end{itemize}

\medskip

As usual, we say that two elements  $a, b :A$ are \emph{propositionally equal} if 
 the type $\Id(a,b)$ is inhabited.
Most work on W-types to date (\emph{e.g.}~\cite{DybjerP:repids,MoerdijkI:weltc,AbbottM:concsp}) has been in the setting of extensional type theories,  
in which the following rule, known as the \emph{identity reflection rule}, is also assumed:

\begin{equation}
\label{equ:collapse}
\begin{prooftree}
 p :  \id{A}(a,b)
  \justifies
  a=b :  A
\end{prooftree}
\end{equation}

This rule collapses propositional equality with definitional equality, thus making the overall system
somewhat simpler to work with. However, it destroys the constructive character of the intensional system, since it makes type-checking undecidable~\cite{HofmannM:extcit}. For this reason, it is not assumed
in the most recent formulations of Martin-L\"of type theories~\cite{NordstromB:marltt} or in automated proof assistants like Coq~\cite{BertotY:inttpp}.


In intensional type theories, inductive types cannot be characterized by standard category-theoretic
universal properties. For instance, in this setting it is not possible to show that there exists a 
definitionally-unique function out of the empty type with rules as in~\cite[Section~5.2]{NordstromB:marltt}, thus making it impossible to prove that the empty type provides an initial object. 
Another consequence of this fact is that, if we attempt to define the type of 
natural numbers as a W-type in the usual way, then 
the usual elimination and computation rules for it are no longer derivable~\cite{DybjerP:repids}. Similarly, it is not possible to show the uniqueness of recursively-defined functions out of W-types. When interpreted categorically, the uniqueness of such functions translates into the initiality property of the associated polynomial functor algebra, which is why the correspondence between W-types and initial algebras fails in the intensional setting.

Due to this sort of poor behaviour of W-types, and other constructions, in the purely intensional setting, that system is often augmented by other extensionality principles that are somewhat weaker than the Reflection rule, such as Streicher's K-rule  or the Uniqueness of Identity Proofs (UIP)~\cite{StreicherT:invitt}, which has recently been reconsidered
in the context of Observational Type Theory \cite{AltenkirchT:obsen}.  Inductive types in such intermediate systems are somewhat better behaved, but still exhibit some undesirable properties, making them less useful for practical purposes than one might wish~\cite{McBrideC:wtygnb}.  Moreover, these intermediate systems seem to lack a clear conceptual basis:  they neither intend to formalize constructive sets (like the extensional theory) nor is there a principled reason to choose these particular extensionality rules, beyond their practical advantages.  

%%%%%%%%%%%%%%%%%%%%%%%%%%%%%%%%
\subsection{The system $\Hint$} 

\noindent We here take a different approach to inductive types in the intensional setting, namely, one motivated by the homotopical interpretation.  It involves working over a dependent type theory $\Hint$ which has the following deduction rules on top of the standard structural rules:
\begin{itemize}
\item rules for identity types as stated above;
\item rules for $\Sigma$-types as in~\cite[Section~5.8]{NordstromB:marltt};
\item rules for $\Pi$-types as in~\cite[Section~3.2]{GarnerR:strdpt}; 
\item the Function Extensionality axiom (FE), \emph{i.e.} the axiom asserting that
for every $f, g : A \rightarrow B$, the type
\[
(\Pi x :  A)\id{B}( \app(f, x), \app(g, x)) \rightarrow \id{A \rightarrow B}(f,g) 
\]
is inhabited.
\end{itemize}
Here, we have used the notation $A \rightarrow B$ to indicate function types, defined via
$\Pi$-types in the usual way. Similarly, we will write $A \times B$ to denote the binary product
of two types as usually defined via $\Sigma$-types.
\smallskip

\subsubsection*{Remarks}
\begin{enumerate}[(i)]
\item The rules for $\Pi$-types of $\Hint$ are derivable from those
in~\cite[Section~5.4]{NordstromB:marltt}. For simplicity, 
we will write~$f(a)$ or~$f  a$ instead of $\app(f,a)$. 
\item As shown in~\cite{VoevodskyV:unifc}, the principle of propositional function extensionality stated above implies
the corresponding principle for dependent functions, \emph{i.e.} 
\[
(\Pi x :  A)\id{B(x)}( f x, g x) \rightarrow \id{(\Pi x : A) B(x)}(f,g) \, .
\]
\item The following form of the $\eta$-rule for $\Sigma$-types is derivable:
\[
\begin{prooftree}
c  : (\Sigma x : A)B(x) 
\justifies
\eta_{\Sigma}(c) : \Id(c, \pair( \pi_1 c \, , \pi_2 c)) \, , 
\end{prooftree}
\]
 where $\pi_1$ and $\pi_2$ are the projections. This  can be proved by $\Sigma$-elimination,
without FE.

\item The following form of the $\eta$-rule for $\Pi$-types is derivable:
\[
\begin{prooftree}
f : (\Pi x : A)B(x)
\justifies
\eta_\Pi(f) : \Id(f , \,  \lambda x. \, f x)
\end{prooftree}
\]
This is an immediate consequence of FE and clearly implies
the corresponding $\eta$-rule for function types.
\item $\Hint$ does \emph{not} include the $\eta$-rules as definitional equalities, either for $\Sigma$-types or for $\Pi$-types (as is done in~\cite{GoguenH:inddtw}).
\item The type theory $\Hint$ will serve as the background theory for our study of 
inductive types and W-types. For this reason, we need not assume it to have any primitive types.
\end{enumerate}


\noindent
This particular combination of rules is motivated by the fact that $\Hint$ has a clear
homotopy-theoretic sematics. Indeed, the type theory~$\Hint$ is a subsystem of the type theory 
used in Voevodsky's Univalent Foundations library~\cite{VoevodskyV:unifc}.  In particular, the 
Function Extensionality axiom is formally implied by Voevodsky's Univalence axiom~\cite{VoevodskyV:notts}, 
which is also valid in homotopy-theoretic models, but will not be needed here. Note that, 
while the Function Extensionality axiom is valid also in set-theoretic models, the Univalence 
axiom is not. Although $\Hint$ has a straightforward set-theoretical semantics, we stress that it 
does not have any global extensionality rules, like the identity reflection rule, K, or UIP. This makes it also compatible with ``higher-dimensional" interpretations such as the groupoid model~\cite{HofmannM:gromtt}, in which the rules of $\Hint$ are also valid.

%%%%%%%%%%%%%%%%%%%%%%%%%%%%%%%%%%%%
\subsection{Homotopical semantics} 

\noindent The homotopical semantics of  $\Hint$ is based on the idea that an identity term~$p:  \id{A}(a,b)$ 
is (interpreted as) a path $p: a\leadsto b$ between the points $a$ and $b$ in the space $A$.   
More generally, the interpretations of terms $a(x)$ and $b(x)$ with free variables will be continuous 
functions into the 
space $A$, and an identity term $p(x) :  \id{A}\big(a(x),b(x)\big)$ is then a 
continuous family of paths, \emph{i.e.}~a homotopy between the continuous functions. Now, the main import of the 
$\Id$-elimination rule is that  type dependency must respect identity, in the following sense: given a dependent type
\begin{equation}
\label{equ:deptype}
x:A \vdash B(x) : \type \, ,
\end{equation} 
and $p: \id{A}(a,b)$, there is then a \emph{transport} function 
 $$p_{\, ! } : B(a) \rightarrow B(b),$$ which is defined by $\Id$-elimination, taking for $x : A$
the function $\refl(x)_{\, !} : B(x) \rightarrow B(x)$ to be the identity on $B(x)$.  Semantically, 
given that an identity term $p: \id{A}(a,b)$ is interpreted as a path $p: a\leadsto b$, 
 this means that a dependent type as in~\eqref{equ:deptype} must be interpreted as a space $B\rightarrow A$, fibered
 over the space $A$,  and that the judgement
  \[
  x,y:A \vdash\id{A}(x,y) : \type
  \] 
  is interpreted as the canonical fibration $A^I \rightarrow A\times A$ 
 of the path space $A^I$ over $A \times A$. For a more detailed overview of the homotopical interpretation, 
 see~\cite{AwodeyS:typth}.

Independently of this interpretation, each type $A$ can be shown to carry the structure of a weak 
$\omega$-groupoid in the sense of~\cite{BataninM:mongcn,LeinsterT:higohc} with the elements of $A$ as objects, identity proofs $p : \id{A}(a,b)$ as morphisms and 
 elements of iterated identity types 
 as~$n$-cells~\cite{vandenBergB:typwg,LumsdaineP:weaci}. Furthermore, $\Hint$ 
 determines a weak $\omega$-category~$\mathcal{C}(\Hint)$ having types as 0-cells, elements $f : A \rightarrow B$ as 1-cells, and elements of (iterated) identity types 
as~$n$-cells~\cite{Lumsdaine:higcft}.   The relation between the weak $\omega$-category structure of~$\mathcal{C}(\Hint)$ and the homotopical interpretation of intensional type theories closely mirrors that between higher category theory and homotopy theory in modern algebraic topology, and some methods developed in the latter setting are also applicable in type theory.  For instance,
 the topological notion of contractibility admits the following type-theoretic counterpart, originally
 introduced by Voevodsky in~\cite{VoevodskyV:unifc}.


\begin{definition}  A type $A$ is called \emph{contractible} if the  type 
 \begin{equation}
 \label{eq:contractible}
\iscontr(A) \defeq (\Sigma x:A)(\Pi y:A)\id{A}(x,y)
\end{equation}
is inhabited.
\end{definition} 

The type $\iscontr(A)$ can be seen as the propositions-as-types translation
of the formula stating that $A$ has a unique element. However, its homotopical interpretation 
is as a space that is inhabited if and only if the space interpreting $A$ is contractible in the usual
topological sense. The notion of contractibility can be used to articulate the world of types into different homotopical dimensions, or \emph{h-levels} \cite{VoevodskyV:unifc}. This classification has proven to be quite useful in understanding intensional type theory.  
For example, it permits the definition of new notions of \emph{proposition} and \emph{set} which provide a useful alternative to the standard approach to formalization of mathematics in type theory~\cite{VoevodskyV:unifc}.

\begin{remark} \label{thm:idcontrcontr}
If $A$ is a contractible type, then for every $a, b : A$, the type $\id{A}(a,b)$ is again contractible. This can be proved  by $\Id$-elimination~\cite{AwodeyS:indtht}. 
\end{remark}

Let us also recall from~\cite{VoevodskyV:unifc} the notions of weak equivalence and homotopy equivalence. To do this, we need to fix some notation. For $f : A \rightarrow B$
and $y : B$, define the type
\[
 \hfiber(f,y) \defeq (\Sigma x : A) \id{B}(f x, y) \, .
\]
We refer to this type as the \emph{homotopy fiber} of $f$ at $y$. 

%%%
\begin{definition} \label{thm:weq} Let $f : A \rightarrow B$.
%%
\begin{itemize}
%
\item We say that $f$ is a \emph{weak equivalence} if  the type
\[
\mathsf{isweq(f)} \defeq (\Pi y : B) \,  \iscontr(\hfiber(f,y)) 
\]
is inhabited. 
%
\item We say that $f$ is a \emph{homotopy equivalence} if there exist a function 
$g : B\rightarrow A$ and elements
\begin{align*}
\eta &: (\Pi x : A) \Id( g  f  x , x) \,  ,\\
\varepsilon &: (\Pi y:B) \Id( f   g  y, y) \, .
\end{align*}
It is an \emph{adjoint homotopy equivalence} if there are also
terms
\begin{align*}
p &: (\Pi x : A) \Id ( \varepsilon_{f x} \, , f \, \eta_x )  \, , \\
q &: (\Pi y : B) \Id ( \eta_{g y} \, , g \, \varepsilon_y) \, ,
\end{align*}
where the same notation for both function application and
the action of a function on an identity proof (which is easily definable by $\Id$-elimination),
and we write $\alpha_x$ instead of $\alpha(x)$ for better readability.
%
\end{itemize}
%%
\end{definition}
%%%

The type $\mathsf{isweq(f)}$ can be seen as the propositions-as-types translation of the formula asserting that $f$ is bijective, while homotopy equivalence is evidently a form of isomorphism. Thus it is a pleasant fact that a function is a weak equivalence if and only if it is a homotopy equivalence~\cite{VoevodskyV:unifc}. 
We also note that all type-theoretic constructions are homotopy invariant, in the sense that they respect this relation of equivalence, a fact which is exploited by the Univalence axiom~\cite{VoevodskyV:notts}.


\medskip

In Section \ref{section:intW} below, these and related homotopy-theoretic insights will be used to study inductive types, but first we must briefly review some basic facts about  inductive types in the extensional setting.


We begin by briefly sketching the basic theory of W-types in the  setting of \emph{extensional} type theory (cf.~\cite{??}), as background for our development; the corresponding details of the intensional case will be provided in what follows.  Let $\mathcal{C}$ be the category with types as objects and elements $f : A \rightarrow B$ as maps, where two maps are considered equal if and only if they are definitionally equal. The premisses of the introduction rule determine the \emph{polynomial endofunctor} $P : \mathcal{C}\rightarrow \mathcal{C}$
defined by 
\[
    P(X) \defeq (\Sigma x : A ) (B(x) \rightarrow X) \, .
\]
A $P$-algebra is a pair consisting of a type $C$ and an arrow $s_C : PC \rightarrow C$, called the structure map of the algebra. A homomorphism of algebras $h : (C, s_C) \to (D, s_D)$ is an arrow $h : C\to D$ making the evident square commute, $h\circ s_C = s_D\circ P(h)$, using the obvious action of $P$ on $h$, vis.\ $P(h)(x,u) = (x, h\circ u)$.  The formation rule gives us an object $$\W \defeq (\W x : A) B(x)$$ and the introduction rule (in combination with the rules for $\Pi$-types
and $\Sigma$-types) provides a structure map $s_\W : P\W \rightarrow \W$.
The elimination rule  states that in order for the projection $\pi_1 \colon E \rightarrow \W$, where
$E \defeq (\Sigma w {\, : \, } \W) E(w)$, to have a section $s$, as in the diagram
\[
\xymatrix{
& E  \ar[d]^{\pi_1} \\
\W \ar[ru]^{s} \ar[r]_{1_\W} & \W, 
}
\]
it is sufficient for the type $E$ to have a $P$-algebra structure over $\W$, i.e.\ for $E$ to have an algebra structure with respect to which $\pi_1$ is a homomorphism. Finally, the computation rule states that the section $\ind$ given by the elimination rule is also a $P$-algebra homomorphism. 

The elimination rule implies what we call the $\W$-recursion rule:
\[
\begin{prooftree}
C : \type  \qquad
 x : A, v : B(x) \rightarrow C \vdash c(x,v) : C
 \justifies
w : \W \vdash \wrecs(w,c) :  C
\end{prooftree}
\]
This can be recognized as a recursion principle for maps from $\W$ into $P$-algebras, since
the premisses of the rule describe exactly a type $C$ equipped with a structure map $s_C 
: PC  \rightarrow C$. For this special case of the elimination rule, the corresponding computation rule states that the function
\[
\lambda w. \wrecs(w,c) : \W \rightarrow C \, ,
\] 
where $c(x,v) = s_C(\pair(x,v))$ for $x : A$ and $v : B(x) \rightarrow C$, is a $P$-algebra homomorphism.
Moreover, this homomorphism can then be shown to be definitionally unique using the elimination rule, the principle of function extensionality and the reflection rule of the extensional system.  

The converse implication also holds: one can derive the general $\W$-elimination rule from the simple elimination rule and the following $\eta$-rule
%
\begin{equation*}
\begin{prooftree}
\begin{array}{l}
C : \type  \qquad w : \W \vdash f(w) : C \\ 
x : A, v : B(x) \rightarrow C \vdash c(x,v) : C\\
x:A \, , u:  B(x) \rightarrow \W  \vdash f\left(\wsup(x,u)) = c(x,\lambda y. fu(y)\right) : C
  \end{array}
 \justifies
w : \W \vdash  f(w) =  \wrecs(w,c) :  C
\end{prooftree}
\end{equation*}
%
stating the uniqueness of the $\wrecs$ term among algebra maps. Since, conversely, the $\eta$-rule can be derived from the the general $\W$-elimination rule, we therefore have that induction and recursion are interderivable in the extensional setting.
%\[
%\begin{array}{ccc}
%\text{\underline{\myemph{Induction}}} & \Leftrightarrow & \text{\underline{\myemph{Recursion}}}\\[1ex]
%\text{$\W$-elimination} & & \text{Simple $\W$-elimination}\\
%\text{$\W$-computation} &&  \text{Simple $\W$-computation + $\eta$-rule} 
%\end{array}
%\]

Finally, observe that induction is equivalent to the statement that, when $\W$ is regarded as a $P$-algebra with the structure map $s_\W : P\W \rightarrow \W$, then every $P$-algebra over it has a homomorphic section, while recursion is equivalent to the statement that 
$\W$ is the initial $P$-algebra. The remainder of this section will be devoted to generalizing these equivalences to the setting of homotopy type theory.

%%%%%%%%%%%%%%%%%%%%%%%%%%%%%%%%


\bigskip

%%%%%%%%%%%%%%%%%%%%%%%%%%
%\subsection{W-types: intensional case}


%Although it is more elaborate to state (and difficult to prove) owing to the presence of 
%recursively generated data, our main result on  W-types is analogous to 
%the foregoing example in the following respect: rather than being strict initial algebras, as in the 
%extensional case, W-types with propositional computation rules are instead homotopy-initial algebras. This fact can again be stated 
%entirely syntactically, as an equivalence between the definability of W-types with propositional computation rules (which we spell
%out below) and the existence of a suitable family of homotopy-initial algebras.  Moreover, as in the simple case of the type $\Bool$ above, the proof is again entirely constructive.
%

%\begin{remark}\label{thm:wtypesinvariance}










%%%%%%%%%%%%%%%%%%%%%%%%%%%%%%%%%%%%%%%%%%%%%%%%%%%%

%
%As before, the above rules imply a non-dependent version of elimination and the corresponding computation rule, where we write $\wrec(m)$ instead of $\wrec^{A,x.B}_{\UU_j}\big(C,x.r.c,m\big)$ where appropriate. \smallskip
%\begin{itemize}
%\item Simple $\W$-elimination rule.\smallskip
%\begin{mathpar}
%\inferrule{A : \UU_i \\ x:A \vdash B(x) : \UU_i \\ m : \W \\ C : \UU_j \\x:A, r : B(x) \to C \vdash c(x,r) : C}{\wrec^{A,x.B}_{\UU_j}\big(C,x.r.c,m\big) : C}
%\end{mathpar}
%\item Simple $\W$-computation rule.\smallskip
%\begin{mathpar}
%\inferrule{A : \UU_i \\ x:A \vdash B(x) : \UU_i \\ a : A \\ t : B(a) \to \W \\ C : \UU_j \\ x:A, r : B(x) \to C \vdash c(x,r) : C}
%{\wrec(\wsup(a,t)) \equiv c(a,\lam{b:B(a)} \wrec(t \;b)) : C}
%\end{mathpar}
%\end{itemize} \smallskip
%
%The induction principle also implies the following uniqueness principles, which state that any two functions out of $\W$ which satisfy the same recurrence are pointwise equal:
%\begin{itemize}
%\item $\W$-uniqueness principle.\smallskip
%\begin{mathpar}
%\inferrule{A : \UU_i \\ x:A \vdash B(x) : \UU_i \\ m : \W \\ w:\W \vdash E(w) : \UU_j \\ x:A, p: B(x) \to \W, r : \prd{b:B(x)} E(p \;b) \vdash
%e(x,p,r) :E(\wsup(x,p)) \\ w:\W \vdash f(w) : E(w) \\ x:A, p: B(x) \to \W \vdash f(\wsup(x,p)) \equiv e(x,p,\lam{b:B(x)} f(p \;b)) \\ w:\W \vdash g(w) : E(w) \\ x:A, p: B(x) \to \W \vdash g(\wsup(x,p)) \equiv e(x,p,\lam{b:B(x)} g(p \;b))}{f(m) \equiv g(m) : E(m)}
%\end{mathpar}
%\item Simple $\W$-uniqueness principle.\smallskip
%\begin{mathpar}
%\inferrule{A : \UU_i \\ x:A \vdash B(x) : \UU_i \\ m : \W \\ C : \UU_j \\ x:A, r : B(x) \to C \vdash c(x,r) : C \\ w:\W\vdash f(w) : C \\ x:A, p: B(x) \to \W \vdash f(\wsup(x,p)) \equiv c(x,\lam{b:B(x)} f(p \;b)) \\ w:\W\vdash g(x) : C \\ x:A, p: B(x) \to \W \vdash g(\wsup(x,p)) \equiv c(x,\lam{b:B(x)} g(p \;b))}{f(m) \equiv g(m) : C}
%\end{mathpar}
%\end{itemize}
%


%%%%%%%%%%%%%%%%%%%%%%%%%%%%%


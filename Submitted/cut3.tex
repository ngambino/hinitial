
\newpage

\begin{proof}
Unfolding We now proceed to find an equivalent, simpler description of the types $G(g, \bar{g}_0, \bar{g}_1)$ and $H(h, \bar{h}_0, \bar{h}_1)$.  


 By $\Id$-elimination, 
we have that~$p^* \langle 1_{a_0}, 1_{a_1} \rangle$ is  propositionally equal to the pair 
\[
\langle 1_{a_0} \ct \alpha_p(a_0)  \, , \;   1_{a_1} \ct \alpha_p(a_1)  \rangle \, , 
\]
where we write $\alpha_p$ for the homotopy associated to the path $p$ via an instance of
the extension function for $\Pi$-types discussed in Section~\eqref{sec:chait}. This pair is propositionally equal 
to the pair $\langle  \alpha_p(a_0), \alpha_p(a_1) \rangle$ by the groupoid laws and 
 therefore, by the characterization of identity types of product types, 
 the type $G(g,\bar{g}_0,\bar{g}_1)$ can be expressed as
\[
(\Sigma p \co \Id( g \com f, 1_A))
\big(
\Id(  
\overline{gf}_0 \, ,  \alpha_p(a_0)) 
\times 
\Id( \overline{gf}_1 \, ,  \alpha_p(a_1))
\big) \, .
\]
Since the expansion functions for $\Pi$-types of Section~\ref{sec:chait} is an equivalence, $G(g, \bar{g}_0, \bar{g}_1)$ is equivalent to
\begin{align*}
(\Sigma \alpha \co \Hot( g \com f, 1_A))
\big( \Id  ( \overline{gf}_0 \, ,  \alpha(a_0) ) 
\times 
\Id ( \overline{gf}_1 \, ,  \alpha(a_1) ) 
\big) \, .
\end{align*}
Analogously, the type $H(h,\bar{h}_0,\bar{h}_1)$ is equivalent to
\[
(\Sigma \beta \co \Hot(f \com h, 1_B))
\big(
\Id  (  \overline{fh}_0 \, ,  \beta(b_0) ) 
\times 
\Id(  \overline{fh}_1 \, ,  \beta(b_1))
\big) \, .
\]
Therefore, we can express $\isbipequiv(f, \bar{f}_0, \bar{f}_1)$ as the type
\begin{multline*} 
(\Sigma g \co B \to  A)
(\Sigma \alpha \co \Hot(g \com f,  1_A))( G_0(g,\alpha) \times G_1(g,\alpha)) \times \\
(\Sigma h \co B \to A)
(\Sigma \beta \co\Hot( f \com h, 1_B)) (H_0(h,\beta) \times H_1(h,\beta)) \, ,
\end{multline*} 
where, for $k \in \{ 0,1 \}$, we set
\begin{align*}
& G_k(g,\alpha) \defeq 
(\Sigma \bar{g}_k \co \Id( g b_k,  a_k))  \, \Id \big( \bar{g}_k \ct g (\bar{f}_k)  \, ,  \alpha(a_k) \big) \, , \\
& H_k(h,\beta) \defeq (\Sigma \bar{h}_k \co \Id( h b_k , a_k)) \, \Id \big(\bar{f}_k \ct f (\bar{h}_k)  \, ,  \beta(b_k) \big)  \, .
\end{align*}
By Lemma~\ref{thm:useful}, we have that $G_k(g,\alpha)$ and
 $H_k(h,\beta)$ are contractible, and therefore we have 
\begin{align*} 
\isbipequiv(f,\bar{f}_0, \bar{f}_1) 
  & \simeq  (\Sigma g \co B \to A)  \, \Hot( g \com f , 1_A ) \times 
(\Sigma h  \co B \to A) \,  \Hot( f \com h, 1_B ) \\
 & \simeq \isequiv(f) \, ,
\end{align*} 
as required.
\end{proof}

\newpage



Unfolding the definition of the type of proofs that $f$ is a $P$-algebra equivalence yields the type
\begin{multline*}
	 (\Sigma g \co D \to C) 
	 \big( \Sigma \bar{g} \co \Id( \sup_C \circ Pg , g \circ \sup_D) \, 
 		G(g, \bar{g}) \; \times
	 \\
	 (\Sigma h \co D \to C) 
	 \big(\Sigma \bar{h} \co  \Id( \sup_C \circ Ph, h \circ \sup_D) 
	 \,  H(h, \bar{h}) \, , 
\end{multline*}
where $G(g, \bar{g})$ and $H(h, \bar{h})$ express that $g$ and $h$ are a left and right inverse for $f$ as 
$P$-algebra morphisms, respectively. For the right inverse, this amounts to requiring that the pasting diagram

is propositionally equal to the diagram for the identity $P$-algebra morphism on $C$ in~\eqref{equ:palgid}. 
Explicitly, we have
\begin{align*}
G(g,\bar{g}) & 
\defeq 
\Id \big( ( g  f,  \overline{gf}), (1_C, \overline{1}_C) \big)  \\
H(h,\bar{h})    & \defeq \Id \big( (   f h,  \overline{ f h}), (1_D, \overline{1}_D) \big)  \, .
\end{align*}
Using the characterization of paths in $\Sigma$-types, the type $G(g,\bar{g})$ can be equivalently expressed as
\begin{align*}
(\Sigma p \co \Id(g f ,  1_C) ) \, 
	\Id\big( \overline{gf}   ,\, p^*( \overline{1}_C)  \big) \,.
\end{align*}
By path induction on $p$, the transport $p^*(\overline{1}_C)$ is propositionally equal to the composite path
\[
\xymatrix@C=1.7cm{
\sup_C \circ P(gf) \ar[r]^{\sup_C \circ P(p)} & \sup_C \circ P(1_C) \ar[r]^{\bar{1}_C} & 1_C \circ \sup_C \ar[r]^{p^{-1} \circ \sup_C} & 
(g f) \circ 1_C }
\]
This is propositionally equal to
\[ 
(\lambda x)
(\lambda u)
\big( 
\ext(p)_{\sup_C(x,u)} \ct \big(\sup_C(x, \ext^{-1}(\ext(p) \circ u))\big)^{-1}
\big) \, .
\]
Using the characterization of identity types of $\Pi$-spaces, the type $G(g,\bar{g})$ is equivalent to
\begin{multline*} 
(\Sigma p \co  \Id( g  f , 1{_C}))
(\Pi x \co A) 
(\Pi u \co B(x) \to  C) \\
	\Id\Big(g(\bar{f}(x,u)) \ct \bar{g}(x, f \com u),\, \ext(p)_{\sup_C(x,u)} \ct \big(\sup_C(x, \ext^{-1}(\ext(p) \circ u))\big)^{-1}\Big).
\end{multline*}
Since the function $\ext$ is an equivalence, this is equivalent to
\[
(\Sigma \alpha \co \Hot( g \circ f, 1{_C}))
(\Pi x \co A) 
(\Pi u \co B(x) \to C) 
	\Id\big( \ext( \overline{gf}) , \,  (\alpha^{-1} \circ \sup_C ) \ct ( \sup_{C} \circ P(\alpha) ) \big) \, .
\]
Analogously, the type $H(h,\bar{h})$ is equivalent to
\[
(\Sigma \beta \co \Hot( f \com h , 1_D))
(\Pi x \co A) 
(\Pi u \co B(x) \to D) 
\Id \Big(f(\bar{h}(x,u)) \ct \bar{f}(x, h \com u),\, \beta(\sup_D(x,u)) \ct \big(\sup_D(x,\ext^{-1}(\beta \circ u))\big)^{-1}\Big).
\]
Therefore, we can express $\mathsf{isalgequiv}(f,\bar{f})$ as the type
\[ 
(\Sigma g \co C\to D) 
(\Sigma \alpha \co \Hot(g \circ f, 1_C)) G'(g,\alpha) \times
(\Sigma h \co D\to C) 
(\Sigma \beta \co \Hot(f \circ h , 1_D)) \, 
	  H'(h,\beta)
	\]
where
\begin{multline*}
G'(g,\alpha) \defeq 
(\Sigma \bar{g} \co 
(\Pi x \co A) 
(\Pi u \co B(x) \to D) \;
\Id(g(\sup_D(x,u)), \sup_C(x,g \circ u)) \\
(\Pi x \co A) 
(\Pi u \co B(x) \to C) \;
		 \Id\Big(g(\bar{f}(x,u)) \ct \bar{g}(x, f \com u),\, \alpha(\sup_C(x,u)) \ct \big(\sup_C(x,\ext^{-1}(\alpha \circ u))\big)^{-1}\Big) 
		\end{multline*}
		and
\begin{multline*}
H'(h,\beta) \defeq 
(\Sigma \bar{h} \co (\Pi x \co A) (\Pi u \co B(x) \to D) \; \Id(h(\sup_D(x,u)), \sup_C(x,h\circ u))) \\
	 (\Pi x \co A) (\Pi u \co B(x) \to D) 
	 		\Id\Big(f(\bar{h}(x,u)) \ct \bar{f}(x, h \com u),\, \beta(\sup_D(x,u)) \ct \big(\sup_D(x,\ext^{-1}(\beta \circ u))\big)^{-1}\Big).
\end{multline*}
Using the type-theoretic choice principle, we can express $H'(h,\beta)$ equivalently as
\begin{multline*}
(\Pi x \co A) 
(\Pi u \co B(x) \to D) 
(\Sigma \bar{h} \co \Id(h(\sup_D(x,u)), \sup_C(x,h \circ u)))  \\ 
\Id\Big(f(\bar{h}) \ct \bar{f}(x, h \com u) ,\, \beta(\sup_D(x,u)) \ct \big(\sup_D(x,\ext^{-1}(\beta \circ u))\big)^{-1}\Big) \, .
\end{multline*} 
Since $f$ is an equivalence,  the type $H'(h,\beta)$ is contractible by Lemma~\ref{thm:useful}.
Moreover, since $f$ is an equivalence between $C$ and $D$, the mapping $u  \mapsto f \com u$ is an equivalence between $B(x) \to C$ and~$B(x) \to D$. The mapping 
$\bar{g} \mapsto (\lambda x) (\lambda u) \bar{g}(x,f \com u)$ is therefore an equivalence between the types 
\[
(\Pi x \co A) 
(\Pi u \co B(x) \to  D) 
\Id(g(\sup_D(x,u)), \sup_C(x,g \com u))
\] and
\[
(\Pi x \co A) 
(\Pi u \co B(x) \to C)
 \Id(g(\sup_D(x,f \com u)), \sup_C(x,g \com f \com u) )\, .
 \] 
Thus, the type $G'(g,\alpha)$ can be expressed equivalently as
\begin{multline*}
(\Sigma \gamma \co 
(\Pi x \co A)
(\Pi u \co B(x) \to C) 
\Id(g(\sup_D(x,f \circ u)),\, \sup_C(x,g \circ f \circ u))) \\ 
(\Pi x \co A) 
(\Pi u \co B(x) \to C) 
	\Id\big(g(\bar{f}(x,u)) \ct \gamma(x,u),\, \alpha(c(x,u)) \ct \big(c(x)(\ext^{-1}(\alpha \circ u))\big)^{-1}\big)
\end{multline*}
Again using the type-theoretic distributivity, we see that this is equivalent to
\begin{multline*}
(\Pi x \co A)
(\Pi u \co B(x) \to C)
(\Sigma \gamma \co \Id(g(d(x,f \circ u)), c(x,g \circ f \circ u))) \\
  \Id \big(g(\bar{f}(x,u)) \ct \gamma , \alpha(\sup_C(x,u)) \ct \big(\sup_C(x, \ext^{-1}(\alpha \circ u))\big)^{-1}\big)
  \end{multline*} 
Therefore, by Lemma~\ref{thm:useful}, we have that also $G'(g,\alpha)$ is contractible.  Therefore,
\begin{align*} 
\isalgequiv(f,\bar{f}) 
 & \simeq (\Sigma g \co D \to C) \, \Hot( g \com f,1_C )\times 
		(\Sigma h  \co D \to C) \, \Hot( f \com h, 1_D)  \\
 & \simeq  \isequiv(f) \, ,
\end{align*} 
as required.

\[
\begin{array}{cc}
\begin{prooftree}
x \co A \vdash E(x) \co \type \qquad 
x \co A \vdash f(x) \co E(x) 
\justifies
x \co A \vdash \eta_f(x) \co  \ind(x, f(a_0), f(a_1) ) = f(x) 
\end{prooftree} &
\textup{($\eta$-rule)} \\[5ex] 
\begin{prooftree}
x \co A \vdash E(x) \co \type \qquad 
x \co A \vdash f(x) \co E(x) 
\justifies
\mathsf{p}_k(f)  \co  \mathsf{comp}_k( f(a_0), f(a_1))  = \eta_f(a_k) 
\end{prooftree} &
\textup{(Coherence)} \\[3ex]
\end{array}
\]
\end{lemma}


\begin{prooftree}
B \co \type \qquad 
x \co A \vdash f(x) \co B 
\justifies
x \co A \vdash \eta_f(x) \co  \ind(x, f(a_0), f(a_1) ) = f(x) 
\end{prooftree} \\ \medskip 
\begin{prooftree}
B \co \type \qquad 
x \co A \vdash f(x) \co B
\justifies
\mathsf{p}_k(f)  \co  \mathsf{comp}_k( f(a_0), f(a_1))  = \eta_f(a_k) 
\end{prooftree} 


\[
\begin{array}{cc}
\begin{prooftree}
B : \type \qquad
b_0 : B \qquad
b_1 : B 
\justifies
x \co A \vdash \rec(x, b_0, b_1) : B
\end{prooftree} & \textup{(Recursion)} \\[5ex]
\begin{prooftree}
B : \type \qquad
b_0 : B \qquad
b_1 : B 
\justifies
\beta_k(b_0, b_1) :  \rec(a_k, b_0, b_1) =  b_k 
\end{prooftree} &
\textup{($\beta$-rule)} \\[5ex]
\begin{prooftree}
B \co \type \qquad 
x \co A \vdash f(x) \co B 
\justifies
x \co A \vdash \eta_f(x) \co  \rec(x, f(a_0), f(a_1) ) = f(x) 
\end{prooftree}  &
\textup{($\eta$-rule)} \\[5ex] 
\begin{prooftree}
B \co \type \qquad  
x \co A \vdash f(x) \co B
\justifies
p_{f,k} \co  \beta_k( f(a_0), f(a_1)) = \eta_f(a_k)  \medskip
\end{prooftree} & 
\textup{(Coherence)}
\end{array} 
\]


Let $A$ be a bipointed type
and assume that it satisfies the rules stated above. We need to show that it is homotopy-initial. So, let $B = (B, b_0, b_1)$ be a bipointed type. We begin to prove
that the type of bipointed morphisms $\Bip(A,B)$ is contractible by showing that it is inhabited. This is easy: the recursion rule allows us to define a function~$(\lambda x \co A) \rec(x, b_0, b_1) \co A \to B$, 
which can be equipped with the structure of a bipointed morphism since the $\beta$-rules give 
us~$\beta_k \co \rec(a_k, b_0, b_1) = b_k$, for $k \in \{ 0, 1\}$, as required. To complete the proof, we now show that for every bipointed morphism $(f, \bar{f}_0, \bar{f}_1) \co A \to B$ we have a path
\[
(f, \bar{f}_0, \bar{f}_1) = 
\big( (\lambda x \co A) \rec(x, b_0, b_1), \beta_0, \beta_1 \big)      \, .
 \]
We begin by proving $f = (\lambda x \co A) \rec(x, b_0, b_1)$. By function extensionality, it suffices to show that, for $x \co A$, we have $ f(x) =  \rec(x, b_0, b_1)$.
But we have paths
\[
\xymatrix@C=1cm{
f(x) \ar[r]^-{\eta_f(x)} & \rec(x, f(a_0), f(a_1)) \ar[r]^-{\gamma_x} &  \rec(x, b_0, b_1) }  \, ,
\] 
where $\gamma_x$ is defined by $\Id$-elimination over $\bar{f}_0$ and $\bar{f}_1$. 
By the properties of transport, it suffices to show that the two paths represented
by composites in the diagram
\[
\xymatrix@C=1.5cm{
\rec(a_0, f(a_0), f(a_1)) \ar[d]_{\eta_f(a_0)} \ar[r]^-{\gamma_{a_0}} & \rec(a_0, b_0, b_1) \ar[d]^{\beta_0(b_0, b_1)} \\
f(a_0) \ar[r]_{\beta_0} & b_0}
\]
are propositionally equal. By the coherence rule, it suffices to show the corresponding
statement for the diagram
\[
\xymatrix@C=1.5cm{
 \rec(a_0, f(a_0), f(a_1))  \ar[d]_{\beta_0( f(a_0), f(a_1))}\ar[r]^-{\gamma_{a_0}} &  \rec(a_0, g(a_0), g(a_1))
 \ar[d]^{\beta_0(b_0, b_1)} \\
 b_0 \ar[r]_{\beta_0} & g(a_0)}
\]
The required propositonal equality now follows by $\Id$-elimination over $\bar{f}_0 \co f(a_0) = b_0$. The final
propositional equality is proved analogously.



\begin{remark}[Extensional type theory]  \label{thm:extbip} In the fully extensional theory $\Hext$, which contains the 
propositional reflection rule, it is easy to prove that a recursive bipointed type  is inductive, i.e.\ the converse of the statement in Proposition~\ref{thm:indrec}. To see this,
observe that if~$A = (A, a_0, a_1)$ is a recursive bipointed type, then the following rules can be derived in
$\Hext$. \\[1ex]

\begin{itemize}
\item Recursion rule: 
\[
\begin{prooftree}
B : \type \qquad
b_0 : B \qquad
b_1 : B 
\justifies
x \co A \vdash \rec(x, b_0, b_1) : B
\end{prooftree} \medskip
\]
\item Judgemental $\beta$-rules: \\[1ex]
\[
\begin{prooftree}
B : \type \qquad
b_0 : B \qquad
b_1 : B 
\justifies
  \rec(a_0, b_0, b_1) \deq b_0 \co B
\end{prooftree}   \qquad
 \begin{prooftree}
B : \type \qquad
b_0 : B \qquad
b_1 : B 
\justifies
  \rec(a_1, b_0, b_1) \deq b_1 \co B
\end{prooftree} \bigskip
\]
\item Judgemental $\eta$-rule: \\[1ex]
\[
\begin{prooftree}
B \co \type \quad 
x \co A \vdash f(x) \co B \\
\justifies
x \co A \vdash  \rec(x, f(a_0), f(a_1) )  \deq f(x)
\end{prooftree}
\]
\end{itemize} \bigskip
In order to derive the elimination rule for an inductive type, we assume its premisses and define
\[
E' \defeq (\Sigma x : A) E(x) \co \type
\]
We can then apply the recursion rule above as follows:
\[
\begin{prooftree}
E' \co \type \qquad
e'_0 \co E' \qquad
e'_1 \co E'
\justifies
x \co A \vdash \rec(x, e'_0, e'_1) \co E' \, ,
\end{prooftree}
\]
where
\[
 e'_0 \defeq \pair(a_0, e_0) \co E' \, , \quad
 e'_1 \defeq \pair(a_1, e_1) \co E' \, .
 \]
For $x \co A$, we  define 
\begin{equation}
\label{equ:indbipdef}
\ind(x, b_0, b_1) \defeq \pi_2 \big(  \rec(x, e'_0, e'_1) \big) \co E\big( \pi_1 ( \rec(x, e'_0, e'_1)) \big) \, .
\end{equation}
Indeed, we have $\ind(x, b_0, b_1) \co E(x)$ as required, since we can show that
\[
\pi_1 ( \rec(x, e'_0, e'_1))  \deq x \co A \, .
\]
This is because both elements are judgementally equal to $\rec(x, a_0, a_1)  \co A$ by the
judgemental~$\eta$-rule. The computation rules follows easily: for example, the first one
can be obtained as follows:
\begin{align*} 
\ind(a_0, b_0, b_1)  & \defeq \pi_2 \big(  \rec(a_0, e'_0, e'_1) \big) \\
 & \defeq \pi_2 (e'_0) \\
 & \defeq \pi_2 (\pair(a_0, b_0)) \\
 & \defeq b_0 \, .
\end{align*}
It should be noted that this argument relies essentially on the fact that the rules for recursive types
imply, in $\Hext$, that a recursive bipointed type $A$ is universal, in the sense that for every bipointed 
type $B$ there exists a unique (up to judgemental equality) homomorphism of bipointed types $f \co A \to B$.
Indeed, this is what we used to show that the term $\ind(x, b_0, b_1)$ defined in~\eqref{equ:indbipdef}, has
the correct type.
\end{remark}


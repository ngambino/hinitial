We briefly recall here the theory of inductive types, particularly W-types, in fully extensional type theories. 
As an especially simple running example, we first consider the type $\Bool$ of Boolean truth values~\cite[Section~5.1]{NordstromB:marltt}. 


\subsection{W-types}\label{subsection:wtypes}

%%%%%%%%%%%%%%%%%%%%%%%%%%%%%%%%%%%%%%%%%%%%%%%%%%%%%%%%%%%%%%%%%%%%%%%%%%%%%%%%%%%%%%%%%%%%%%%%%%%%%%%%%%
%%%%%%%%%%%%%%%%%%%%%%%%%%%%%%%%%%%%%%%%%%%%%%%%%%%%%%%%%%%%%%%%%%%%%%%%%%%%%%%%%%%%%%%%%%%%%%%%%%%%%%%%%%

%%%%%%%%%%%%%%


%\subsection{Inductive types as W-types}
%
%\noindent To conclude our review, recall that in extensional type theory, many inductive types can be reduced to W-types.  We mention the following  examples, among many others (see \cite{MartinLofP:inttt}, \cite{DybjerP:repids}, \cite{GoguenH:inddtw}, \cite{MoerdijkI:weltc}, \cite{GambinoN:weltdp}, \cite{AbbottM:concsp}):
%\begin{enumerate}
%\item \emph{Natural numbers}. \label{extnatW}
%The usual rules for $\nat$ as an inductive type can be derived from its formalization as the following W-type. Consider the signature with two operations, one of which has arity $0$ and one of which has arity $1$; it is presented type-theoretically by a dependent type with corresponding polynomial functor (naturally isomorphic to)
%\[
%P(X) = \mathsf{1} + X \, ,
%\]
%%
%and the natural numbers $\nat$ together with the canonical element $0:\nat$ and the successor function $s : \nat\rightarrow\nat$ form an initial $P$-algebra
%\[
%(0, s) : \mathsf{1} + \nat \rightarrow \nat\, .
%\]
%%
%\item \emph{Second number class.}
%As shown in~\cite{MartinLofP:inttt}, the second number class can be obtained as a W-type determined by the polynomial functor 
%\[
%P(X) = \mathsf{1} + X + (\nat \rightarrow X) \, .
%\]
%This has algebras with three operations, one of arity $0$, one of arity $1$, and one of arity (the cardinality of) $\nat$.
%%
%%\item \emph{Lists.}  The type $\List(A)$ of finite lists of elements of type $A$ can be built as a W-type determined by the polynomial functor 
%%\[
%%P(X) = \mathsf{1} + A\!\times\! X \, .
%%\]
%%We refer to \cite{xxx} % need a reference here !
%% for details.
%\end{enumerate}
%
%\smallskip